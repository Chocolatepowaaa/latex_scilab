\documentclass[a4paper,12pt]{article}

\usepackage[utf8x]{inputenc}
\usepackage[T1]{fontenc}
\usepackage[francais]{babel}
\usepackage{amsmath,amssymb}
\usepackage{fullpage}
\usepackage{xspace}
\usepackage{verbatim}
\usepackage{graphicx}
\usepackage{listings}
\usepackage[usenames,dvipsnames]{color}
\usepackage{url}
\usepackage{subfig}

\captionsetup[subfloat]{%
  labelformat=empty}

\lstset{basicstyle=\small\tt,
  keywordstyle=\bfseries\color{Orchid},
  stringstyle=\it\color{Tan},
  commentstyle=\it\color{LimeGreen},
  showstringspaces=false}

\newtheorem{question}{Question}
\newtheorem{exo}{Exercice}

\newcommand{\dx}{\,dx}
\newcommand{\ito}{,\dotsc,}
\newcommand{\R}{\mathbb{R}}
\newcommand{\C}{\mathbb{C}}
\newcommand{\N}{\mathbb{N}}
\newcommand{\Poly}[1]{\mathcal{P}_{#1}}
\newcommand{\abs}[1]{\left\lvert#1\right\rvert}
\newcommand{\norm}[1]{\left\lVert#1\right\rVert}
\newcommand{\pars}[1]{\left(#1\right)}
\newcommand{\bigpars}[1]{\bigl(#1\bigr)}
\newcommand{\set}[1]{\left\{#1\right\}}

\title{Compte rendu de TP Mini-stage Scilab/latex}
\author{Alexandre Petit \and Térence Marchal \and Olivier brun }
\date{Pour le 22 Novembre 2017}

% ===============
%\begin{document}
% ===============

\begin{document}
\maketitle
\section{ Sensibilisation à l'arithmétique machine}
\subsection{ Exercice 1}

\subsubsection{Programme de test}
\begin{verbatim}

  --> x = 1e30
  --> y = 1e-8
  --> z = ((y + x) - x) / y
  --> w = (y + (x - x)) / y
\end{verbatim}

\subsubsection{Analyse des résultats}
On rappelle que dans scilab, les nombres réels sont représentés par un nombre flottant dont la mantisse est codée sur $52$ bits et l'exposant sur $11$ bits. On remarque que $2^{52} \approx 4 \times 10^{15}$, en d'autres termes, cette représentation permet d'atteindre approximativement $16$ chiffres représentatifs (décimaux). \\\\
Dans le calcul de $z$, l'opération $y + x$ donne, en notation décimale
$$
  x + y = 1000000000000000000000000000000,00000001
$$
\newline
Cette valeur présente $39$ chiffres significatifs, ce qui est bien au delà des capacités de repésentation en double précision.\\\\
De fait, dans $x + y$, $y$ se trouve absorbé (par sa faible pondération), et l'on obtient $x + y = x$ bien que $y$ soit non nul.\\\\
Dans le calcul de $w$, l'opération $x - x$ produit un zéro, calculer $w$ revient alors à diviser $y$ par lui même, ce qui donne $1$, le calcul est correct.

\subsection{ Exercice 2}

Pour cet exercice, deux implantations des calculs sont proposées. Les résultats sont identiques par chaque méthode. La première est immédiate. La seconde, récursive, permet de suivre l'évolution des calculs pas à pas.\\\\
Posons
\begin{itemize}
  \item $recursive\_root(x, n) = \sqrt{\sqrt{\sqrt {\dots{\sqrt{ x}}}}}$ ($n$ fois)
  \item $recursive\_power(x, n) = ((\dots((y^2)^2)^2\dots)^2)^2$ ($n$ fois)
\end{itemize}
L'étude de limite de la fonction racine récursive donne :
$$
\lim\limits_{n \to +\infty} recursive\_root(x, n) = \left\{
    \begin{array}{ll}
        0 & \text{si} \quad x \in \{0, 1\} \\
        1 & \text{si} \quad x \in [0, 1]
    \end{array}
\right.
$$
L'étude de limite de la fonction puissance récursive donne :
$$
\lim\limits_{n \to +\infty} recursive\_power(x, n) = \left\{
    \begin{array}{ll}
        0 & \text{si} \quad x < 1 \\
        1 & \text{si} \quad x = 1 \\
        \infty & \text{si} \quad x \in > 1
    \end{array}
\right.
$$


Au préalable, on pose
\begin{verbatim}
  --> x = 1:0.1:4
\end{verbatim}

\subsubsection{Méthode 1}
\begin{verbatim}
  --> y = x ** (0.5 ** 128)
  --> z = y ** (2.0 ** 128)
\end{verbatim}

\subsubsection{Méthode 2}
\begin{verbatim}
      function y = recursive_root(x, n)
          if (n == 0) then
              y = 1
          elseif (n == 1) then
              y = sqrt(x)
          else
              y = sqrt(recursive_root(x, n - 1))
          end
          disp(y)
      endfunction

      function y = recursive_power(x, n)
          if (n == 0) then
              y = x;
          else
              z = recursive_power(x, n - 1)
              y = z .* z;
          end
          disp(y)
      endfunction

  --> y = recursive_root(x, 128)
  --> z = recursive_power(y, 128)
\end{verbatim}

\subsubsection{Représentation graphique}
Tracé des courbes pour $x$ allant de $0$ à $4$ avec un pas de $0.1$.

\begin{figure}[h]
    \centering
    \includegraphics[width=\textwidth]{recursive_root_x}
\end{figure}

\subsubsection{Analyse des résultats}

Afin de visualiser les étapes de calcul, lançons la commande $recursive\_root(x, n)$ avec la valeur scalaire $x = 4$ au rang $n = 128$.

\begin{verbatim}
--> recursive_root(4, 128)
    2.
    1.4142136
    1.1892071
    1.0905077
    1.0442738
    1.0218971
    1.0108893
    1.0054299
    1.0027113
    1.0013547
    1.0006771
    1.0003385
    1.0001692
    1.0000846
    1.0000423
    1.0000212
    1.0000106
    1.0000053
    1.0000026
    1.0000013
    1.0000007
    1.0000003
    1.0000002
    1.0000001
    1.
    1.
    1.
    1.
    1.
    1.
    ...
\end{verbatim}

On observe qu'au fil des itérations, la fonction tends vers $1$ jusqu'à venir s'écraser sur cette valeur (ici lors de la 26e itération). Autrement dit, la valeur est tellement proche de $1$ que la machine ne peut plus représenter : Soit $y_i$ la valeur prise lors de la $i^{ième}$ itération, on a :

$$
1 < y_i < 1 + 10^{-16} \qquad \forall i = 26,\dots,128
$$

// Impossible à redresser / inverser


\subsection{ Exercice 3}
On rappelle la définition de l'intégrale étudiée dans cet exercice :
$$
I_n=\int _0^1x^ne^xdx
$$
\subsubsection{}

On donne une formule de récurrence de $I_n$, pour tout $n \geq 0$
$$
  \left\{ \begin{aligned}
    I_0&=e-1\\
    I_{n+1}&=e-(n+1)I_n\\
  \end{aligned} \right.
$$

\subsubsection{}
Effectuons le développement en série de l'intégrale :
\begin{align*}
  I_{20}
  &=\int_0^1 x^n \sum_{k=0}^ \infty \frac{x^k}{k!}dx\\
  &=\sum_{k=0}^\infty \int_0^1\frac{x^{k+20}}{k!}dx\\
  &=\sum_{k=0}^\infty\frac{1}{k!(k+21)}
\end{align*}\newline

D'où $I_{20}\approx \sum_{k=0}{N}\frac{1}{k!(k+21)}$ pour N suffisamment grand.

\subsection{Exercice 4}

Méthode des rectangles :\newline
$$
\int_0^1f(t)dt\approx\frac{1}{n-1}\sum_{k=0}^{n-1}f(k)
$$
\newline
On applique à $f:x\mapsto x^{20}e^x$ avec SCILAB.

\section{ Etude du phénomène de Gibbs}

\subsection{ Exercice 5}
On étudie une fonction $f: \mathbb{R} \rightarrow \mathbb{R}$ de période $T = 1$. En vue de calculer la série de Fourier de $f$, on calcul les coefficients $a_n(f)$.
\begin{align*}
  a_0(f)
  &=\int_{-\frac{1}{2}}^{\frac{1}{2}}f(t).dt\\
  &=\int_{-\frac{1}{2}}^{0}f(t).dt + \int_{0}^{\frac{1}{2}}f(t).dt\\
  &=\int_{-\frac{1}{2}}^{0}(-1).dt + \int_{0}^{\frac{1}{2}}1.dt \\
  &= 0
\end{align*}\\\\
De par la parité de la fonction cosinus, on a :
\begin{align*}
  a_n(f)
  &=2\int_{-\frac{1}{2}}^{\frac{1}{2}}f(t)\cos(2\pi nt).dt\\
  &=0
\end{align*}
Enfin,
\begin{align*}
  b_n(f)
  &=2\int_{-\frac{1}{2}}^{\frac{1}{2}}f(t)\sin(2\pi nt)dt\\
  &=4\int_0^{\frac{1}{2}}\sin(2\pi nt)dt\\
  &=2\left[\frac{\cos(2\pi nt)}{\pi n}\right]_{\frac{1}{2}}^0\\
  &=2\times\frac{1 - \cos(n\pi)}{\pi n}\\
\end{align*}
La fonction $f$ n'est pas continue en $0$. Autrement, pour tout $x \in \left]-\frac{1}{2};\frac{1}{2}\right[\setminus\{0\}$, les calculs préliminaires des coeffcients nous donnent
\begin{align*}
  f(x)
  &\overset{def}{=}a_{0}(f) + \sum_{n=1}^{\infty} a_{n}(f)\cos\left(2\pi\frac{n}{T}t\right) + b_{n}(f)\sin\left(2\pi\frac{n}{T}t\right)\\
  &=\sum_{n=1}^\infty2\times\frac{1 - \cos(n\pi)}{\pi n}\sin(2\pi nx)\\
  &=\frac{4}{\pi}\sum_{n=0}^\infty\frac{\sin(2\pi nx(2n+1))}{2n+1}
\end{align*}


\newpage
\subsection{ Exercice 6}

On calcul la somme partielle de la fonction $f$ obtenue dans l'exercice précédent.

\begin{figure}[h]
    \centering
    \subfloat[n = 1]{
      \includegraphics[width=0.5\textwidth]{fourrier_n1}
    }
    \subfloat[n = 5]{
      \includegraphics[width=0.5\textwidth]{fourrier_n5}
    }
    \hspace{0mm}
    \subfloat[n = 50]{
      \includegraphics[width=0.5\textwidth]{fourrier_n50}
    }
    \subfloat[n = 5000]{
      \includegraphics[width=0.5\textwidth]{fourrier_n5000}
    }
\end{figure}

On remarque que la transformée de fourier de $f$ approche fortement $f$ lorsque $n$ est suffisament grand.

\section{ Théorème de Gerschgörin}

\subsection{ Exercice 7}
\subsubsection{Démonstration du théorème}
%\begin{equation}
Soit $A = (a_{ij})_{1 \leqslant i, j \leqslant N} \in \mathcal{M}_N(\mathbb{C}),
\lambda \in Sp(A)$  et $ v = (v_i)_{1 \leqslant i \leqslant N}^\intercal$ un vecteur propre de A
 associé à $\lambda$.
%\end{equation}
 On a, pour tout $k$:
 \begin{equation}
     \lambda v_k = \sum\limits_{i=1}^N a_{ik}v_i
 \end{equation}
 Soit:
 \begin{equation}
     (\lambda - a_{kk})v_k = \sum\limits_{i=1, i \neq k}^N a_{ik}v_i
 \end{equation}
 D'où (inégalité triangulaire):
 \begin{equation}
    |\lambda - a_{kk}||v_k| \leqslant \sum\limits_{i=1, i \neq k}^N |a_{ik}||v_i|
 \end{equation}
 Prenons alors k tel que $v_k = \max_{i} |v_i| \neq 0$ car $v$ est un vecteur propre.
 Il vient alors, puisque $v_i \leqslant v_k$ pour tout i:
 \begin{equation}
     |\lambda - a_{kk}||v_k| \leqslant |v_k|\sum\limits_{i=1, i \neq k}^N |a_{ik}|
 \end{equation}
D'où le résultat.

\subsubsection{Visualisation des disques de Gerschgörin}

Considérons la matrice suivante
\begin{equation*}
    A = \left( \begin{array}{cccc}
      1+i & i & 2 \\
      3 & 2+i & 1\\
      1 & i   & 6
    \end{array} \right).
\end{equation*}
et traçons les disques de Gerschgörin qui lui sont associés.
\begin{figure}[h]
    \centering
    \includegraphics[width=\textwidth]{gershgorin}
\end{figure}

\subsubsection{Vérification de l'appartenance des valeurs propres au disques}
On utilise la commande $spec(A)$ pour connaitre ses valeurs propres.
\newline
On constate graphiquement qu'elles sont bien dans les disques
\newline\newline
4. Soit $A$ à diagonale strictement dominante
\newline
En effet, posons $E_k = \{z \in \mathbb{C}, |z - a_{kk}| < |a_{kk}| \}$
 ,il est clair que $0_{\mathbb{C}}\notin\bigcup\limits_{k=1}^N E_k$
\newline
On a également $D_k\subseteq E_k$ pour tout k
\newline
Par suite: $Sp(A)\subseteq\bigcup\limits_{k=1}^N D_k\subseteq\bigcup\limits_{k=1}^N E_k$
donc $0_{\mathbb{C}}\notin Sp(A)$
\newline
Donc $A$ est inversible.
\end{document}
